\documentclass[11pt]{article}
\usepackage[spanish]{babel}
\usepackage[utf8]{inputenc}
\usepackage{booktabs}  
\usepackage{graphicx} 
\usepackage{listings}
\usepackage{amsmath,amsthm,amssymb}
\usepackage{mathtools}
\usepackage{float}

\usepackage
[
  a4paper,% other options: a3paper, a5paper, etc
  left=1.7cm,
  right=2cm,
  top=3cm,
  bottom=4cm,
]
{geometry}

\lstset{%
backgroundcolor=\color{cyan!10},
basicstyle=\ttfamily,
numbers=left,numberstyle=\scriptsize
}

\setlength{\parskip}{\baselineskip}%
\setlength{\parindent}{0pt}%

%\usepackage[wby]{callouts}
\usepackage{hyperref}
\usepackage{url}

% pseudocode
\usepackage{algpseudocode}
\usepackage{algorithm}
\usepackage{algorithmicx}

\usepackage{graphicx}
\usepackage{caption}
\usepackage{subcaption}

\title{Whale Optimziation Algorithm aplicado al problema de clustering}
\author{Yábir García Benchakhtir}
\begin{document}

\maketitle

\begin{figure}[h]
\includegraphics[scale=0.3]{UGR}
\centering
\end{figure}

\newpage

\renewcommand*\contentsname{Índice}
\tableofcontents

\newpage

\section{Introducción al problema}


Consideramos el problema del agrupamiento con restricciones. En este
problema contamos con un conjunto no vacio $P \subset \mathbb{R}^n$ de
puntos y nos planteamos cómo podríamos agruparlos de manera que exista
una relación entre los puntos de un mismo grupo. A cada grupo lo
denominaremos \textit{cluster} y notaremos $\mathcal{C}$ al conjunto
de todos los clusters.

Sobre esta base imponenmos restricciones en la manera en la que se
realizan las agrupaciones. En primer lugar existe un subconjunto de
pares de puntos $ML$ definido como

\[ ML = \{(a,b) \in P\times P\ | \quad a \in K \iff b\in K \text{ para } K \in \mathcal{C}\}\]

es decir, el conjunto de puntos que han de estar en el mismo
cluster. De manera similar existe otro conjunto $CL$ de pares de
puntos que no pueden pertenecer al mismo conjunto.

\[ CL = \{(a,b) \in P\times P\ | \quad a \in K \iff b\notin K \text{ para } K \in \mathcal{C}\}\]

Notaremos por $R = ML \cup CL$ al conjunto de restricciones del problema.

Nos concentraremos en encontrar soluciones bajo restricciones \textit{débiles} a
este problema donde intentaremos encontrar soluciones que minimicen el conjunto
restricciones violadas.

En nuestros algoritmos intentaremos minimizar la distancia que haya
entre los puntos de cada cluster sujeto a que se violen el menor
número de restricciones. Vamos a formalizar pues esta idea,
definimos el centroide de un cluster como el punto promedio de los
puntos del cluster

\[
  \mu_i = \frac{1}{|c_i|}\sum_{x\in c_i}x \quad \text{ con } c_i \in \mathcal{C} \text{ para todo } i \in \{1,...k\}
\]

Definimos la distancia \textit{intra-cluster} como la media de las
distancias de cada punto del cluster al centroide. En este caso
consideramos la distancia euclidea.

\[
  \bar c_i = \frac{1}{|c_i|}\sum_{x\in c_i}||x-\mu_i||_2 \quad \text{ con } c_i \in \mathcal{C} \text{ para todo } i \in \{1,...k\}
\]

La desviación general del problema será por tanto

\[
  \bar C = \frac{1}{k}\sum_{c_i\in \mathcal{C}}\bar c_i
\]

Nuestro objetivo cuanto busquemos una solución con restricciones
débiles será minimizar la desviación general sujeto a que "no
incumplamos demasiadas restricciones". Para obtener un valor que le de
mayor o menor relevancia a la cantidad de restricciones que
incumplamos usaremos la siguiente proporción

\[
  \lambda = \frac{\lceil d \rceil}{|R|}
\]

donde $d$ es la mayor distancia entre puntos del conjunto de puntos
$P$.  Así nuestro objetivo,cuando busquemos una solución con
restricciones débiles, será minimizar

\[
  f = \bar C + \lambda * \text{infeasibility}
\]

donde \textit{infeasibility} es la cantidad de restricciones incumplidas.

\section{Whale Optimziation Algorithm}

La metaheurística elegida para resolver el problema ha sido Whale Optimziation
Algorithm (WOA) propuesta por Seyedali Mirjalili y Andrew Lewis en 2016. Esta
metaheurística basa su comportamiento en las técnicas depredadoras de la ballena
jorobada y su comportamiento social. 

Utilizando el especial comportamiento que tiene este animal cuando colabora con 
otros de su misma especia se pretende conseguir una metaheurística que proporcione
buenos resultados en problemas de optimización de funciones reales intentando 
preservar un equilibrio entre exploración y explotación.

Más concretamente la técnica de caza consiste en crear una espiral entorno a la 
presa y levantar un \textit{muro} de burbujas de aire, haciendo que esta se desoriente
para posteriormente acercarse y atacar. 

\subsection{Descripción de la metaheurística}

Para modelar el problema los autores de la metaheurística proponen un modelado 
matemático del comportamiento de la ballena jorobada que se pueda adaptar adaptar 
a la optimización de una función real.

Los agentes $X$ que participan en nuestro algoritmo (y que representan a las
ballenas) se van a representar como $N$ vectores de dimensión $d$ donde cada
componente del vector del agente representa una coordenada en nuestra solución

\[
    X_i(t) = <X_{i1}(t), X_{i2}(t), \dots, X_{id}(t)>
\]

como deja entrever esta notación el estado de la ballena depende de una variable 
temporal $t$ que represneta el instante de tiempo en el que nos encontramos y 
que está limitado por una constante $T$ que fijamos nosotros. En cada instante 
$t$ la mejor solución vendrá represnetada por $X^*$.

Procedemos en primer lugar a definir el operador distancia entre dos ballenas

\begin{align*}
    ||\cdot||\colon & \mathbb{R}^n\times\mathbb{R}^n \to \mathbb{R}^n\\
    & (<x_1, x_2, \dots, x_n>, <y_1, y_2, \dots, y_n>) \xmapsto{}<|x_1 - y_1|, |x_2-y_2|, \dots, |x_n-y_n|>
\end{align*}

que es la distancia coordenada a coordenada.

El movimiento de caza se represneta por la modificación del vector agente
mediante la expresión

\begin{align}
    X_i(t+1) = X^*(t) - A\cdot D_i^1  \label{eq:1}
\end{align}

donde $\cdot$ representa el producto componente a componente de $\mathbb{R}^n$, $D$ viene dado por
la expresión

\[
    D^1_i = ||CX^*(t)-X_i(t)||    
\]

y $A \text{ y } C$ se obtinen como 

\begin{align*}
    A &= 2a\cdot r - a \\
    C &= 2r
\end{align*}

con $r \in [0,1]^d$ un vector aleatorio y $a\in [0,2]^d$ constante que se hace
decrecer de manera lineal a lo largo de los distintos pasos del algoritmo
mediante la ecuación

\[
  a(t) = 2-2\frac{t}{\text{max\_evaluaciones}}  
\]

La técnica de \textit{caza} se basa en combinar este movimiento que nos
proporciona un componente de exploración junto a la creación de una espiral 
mediante la actualización del agente de acuerdo a la expresión 

\begin{align}
\begin{cases}
    X_i(t+1) &= e^{bl}cos(2\pi l)D_2 + X^* \\
    D^2_i &= ||X^*(t)-X_i(t)||\\
\end{cases}\label{eq:2}
\end{align}

siendo $b \in \mathbb{R}$ constante y $l \in [-1,1]$ aleatorio de manera que
nos definen un radio para la espiral en cada instante. La notación $\oplus$ 
se define como 



El movimiento de caza natural mezcla tanto los desplazamientos en linea recta
como el comportamiento en espiral por lo que se introduce un factor aleatorio
$p \in [0,1]$ que decide que tipo de movimeinto se va a realizar

\[
    X_i(t+1) =  
\begin{cases}
    X^*(t) - A\cdot D^1_i & p < \frac{1}{2}\\
    e^{bl}cos(2\pi l)D^2_i + X^*  & p\geq \frac{1}{2}\\  
\end{cases}
\]

\subsection{Adaptación de la metaheurística al problema}

Durante el desarrollo de la asignatura hemos trabajado con una representación
de la solución que se centraba en la asignación de cada punto a un cluster
y se recalculaba en cada caso los centros de cada cluster. El espacio de búsqueda 
era 

\[
    \{<x_1, x_2, \dots x_n >: x_i \in [0, k] \cap \mathbb{N}, k > 0\}
\]

donde $n$ representa el número de puntos que intervienen en el problema y k el número 
de clusters que consideramos.

Para adaptar la metaheurística he decidido variar mi enfoque del problema y, en
lugar de modificar las asignaciones que hago de los puntos, pensar que cada
agente representa las coordenas de los centroiedes del problema.

Así cada agente (ballena) queda definido como 

\[
    X_i = <<c_{01},c_{02}, \dots c_{0d}>, \overset{(k)}{\dots}, <c_{k1},c_{k2},\dots c_{kd}>>    
\]

una lista de $k$ vectores con dimensión $d$, la dimensión de los puntos del
problema. No obstante esto no es suficiente ya que nuestro objetivo final es 
proporcionar una asignación de los puntos a cada cluster. 

El algoritmo WOA se encargará de minimizar la componente real y continua de la
métrica que estamos evaluando (que además tiene más peso en la misma), la
distancia de los centroides respecto a los puntos de cada cluster. La segunda
componente de la métrica es el factor de \textit{infeasibility} que
minimizaremos asignando a cada punto el cluster que menos incremento en
infeasibility produzca y, en caso de empate, el cluster más cercano.

El comportamiento del algoritmo, como se muestra a continuación, es sencillo 

\begin{algorithm}[H]
  \caption{Whale optimization Algorithm}
  \label{alg-lsearch}
  \begin{algorithmic}[1]
    \Procedure{WOA}{max\_evaluaciones}
    \State whales: Inicializar un conjunto de ballenas con k centroides aleatorios
    \State Evaluamos las diferentes ballenas usando nuestra métrica 
    \State Selecccionamos $X^*$ la mejor ballena
    \State Guardamos la mejor solución encontrada al problema
    \State evaluaciones $\leftarrow$ 0
    \While {evaluaciones $<$ max\_evaluaciones }
    \State actualizamos el parametro $a$
    \For{agente en la lista de ballenas}
    \State Determianr $p$ y calcular  $A$ y $C$.
    \If{$p<0.5$}
    \If {$|A| < 1$}
    \State Movemos la ballena usando el movimeinto rectilineo \eqref{eq:1}
    \ElsIf {$|A|>1$} 
    \State Movimiento rectilineo usando una ballena eleatorio
    \EndIf
    \Else
    \State Movimiento en espiral utilizando \eqref{eq:2}
    \EndIf
    \EndFor 
    \State Comprobar si alguna solución se ha salido de los limites del problema
    \State Incremenetar evaluaciones en el número de evaluaciones correspondiente
    \State Evaluar las soluciones encontradas y actualizar la mejor ballena 
    \If{la mejor ballena es la mejor solución encontrada}
    \State Actualizar la mejor solución encontrada hasta el momento
    \EndIf
    \EndWhile
    \State
    \Return Construir la solución asociada a la mejor ballena encontrada
    \EndProcedure
  \end{algorithmic}
\end{algorithm}

El algoritmo que, a partir de una ballena, nos permite crear una solución en el
formato que hemos descrito es 


\begin{algorithm}[H]
  \caption{Generar una solución a partir de un conjunto de centroides}
  \label{alg-lsearch}
  \begin{algorithmic}[1]
    \Procedure{Generar solución}{datos, restricciones, centroides}
    \State clusters $\leftarrow [[]\dots[]]$  Lista con los puntos asociados a cada cluster
    \For{cada punto $p$ del conjunto de datos}
    \State Calcular el coste de infeasibility que supone asignar el punto a los diferentes clusters 
    \State elegir $K = [k_0,\dots,k_s]$ los clusters que menor incremento de infeasibility suponen
    \If {$K.size() == 1$}
    \State Asignar $p$ al cluster $K[0]$
    \Else
    \State Calcular la dinstancia del punto $p$ al centro de cada cluster $k_i \in K$
    \State Asignar $p$ al cluster más cercano
    \EndIf
    \EndFor
    \State
    \Return Construir la solución asociada a la mejor ballena encontrada
    \EndProcedure
  \end{algorithmic}
\end{algorithm}

\section{Implementación y desarrollo del experimento}

La implementación del código ha sido realizada en el lenguaje de programación
\textit{Rust} ya que ha sido el el usado durante el desarrollo de las anteriores
prácticas, por lo que se ha podido reutilzar código y las ejecuciones han sido
realizadas bajo las mismas optimzaciones en el lenguaje.

Los experimentos han sido ejecutados en un ordenador con las siguientes
características

\begin{itemize}
  \item CPU: Ryzen 5 2600 3.4GHz
  \item RAM: 24GB
  \item SO: Solus OS
\end{itemize}

Cada experimento consistente en la ejecución del algoritmo bajo 5 semillas
distintas (37,42,440,699,752) de las que se han recopilado datos y se han
agreado tomando medias. También se proporcionan en detalle los resultados para
cada ejecución.

El tamaño de la población como hemos estudiado para este tipo de problemas se
recomienda entre 20 y 50 individuos. En mi caso he optado por fijar este valor a
30 individuos a fin de obtener una primera idea del funcionamiento del
algoritmo. El parametro $b$ de la espiral lo he fijado a 1 ya que es el valor
que se fija en la versión implementada en MATLAB. Tras haber
ejecutado el algoritmo los resultados son los siguientes.

\begin{figure}[H]
  \begin{subfigure}{.5\textwidth}
    \centering
    % include first image
    \begin{table}[H]
      \centering
      \resizebox{\textwidth}{!}{
        \begin{tabular}{l|l|l|l|l|}
\cline{2-5}
& \multicolumn{1}{c|}{Iris} & \multicolumn{1}{c|}{Ecoli} & \multicolumn{1}{c|}{Rand} & \multicolumn{1}{c|}{Tyroid}  \\ \hline 
\multicolumn{1}{|l|}{Semilla 37 } &  0.670304  & 13.046238   &  0.715599   &  14.013687 \\ \hline
\multicolumn{1}{|l|}{Semilla 42 } &  0.671846  & 13.140929   &  0.715599   &  13.500604 \\ \hline
\multicolumn{1}{|l|}{Semilla 440 } &  0.656589  & 13.058654   &  0.715599   &  13.295182 \\ \hline
\multicolumn{1}{|l|}{Semilla 699 } &  0.668911  & 13.050045   &  0.715599   &  13.647765 \\ \hline
\multicolumn{1}{|l|}{Semilla 752 } &  0.666602  & 13.026891   &  0.715599   &  13.790620 \\ \hline
\multicolumn{1}{|l|}{ Media } &  0.666851  & 13.064551   &  0.715599   &  13.649572 \\ \hline
\end{tabular}
      }                
    \end{table} 
    \caption{Distancia intracluster}
  \end{subfigure}
  \begin{subfigure}{.5\textwidth}
    \centering
    % include second image
    \begin{table}[H]
      \centering
      \resizebox{\textwidth}{!}{
        \begin{tabular}{l|l|l|l|l|}
\cline{2-5}
& \multicolumn{1}{c|}{Iris} & \multicolumn{1}{c|}{Ecoli} & \multicolumn{1}{c|}{Rand} & \multicolumn{1}{c|}{Tyroid}  \\ \hline 
\multicolumn{1}{|l|}{Semilla 37 } &  42.000000  & 1273.000000   &  0.000000   &  185.000000 \\ \hline
\multicolumn{1}{|l|}{Semilla 42 } &  32.000000  & 1285.000000   &  0.000000   &  204.000000 \\ \hline
\multicolumn{1}{|l|}{Semilla 440 } &  39.000000  & 1298.000000   &  0.000000   &  203.000000 \\ \hline
\multicolumn{1}{|l|}{Semilla 699 } &  48.000000  & 1288.000000   &  0.000000   &  206.000000 \\ \hline
\multicolumn{1}{|l|}{Semilla 752 } &  40.000000  & 1304.000000   &  0.000000   &  191.000000 \\ \hline
\multicolumn{1}{|l|}{ Media } &  40.200000  & 1289.600000   &  0.000000   &  197.800000 \\ \hline
\end{tabular}
      }                
    \end{table}  
    \caption{Infeasibility}
  \end{subfigure}

  \begin{subfigure}{.5\textwidth}
    \centering
    \begin{table}[H]
      \centering
      \resizebox{\textwidth}{!}{
        \begin{tabular}{l|l|l|l|l|}
\cline{2-5}
& \multicolumn{1}{c|}{Iris} & \multicolumn{1}{c|}{Ecoli} & \multicolumn{1}{c|}{Rand} & \multicolumn{1}{c|}{Tyroid}  \\ \hline 
\multicolumn{1}{|l|}{Semilla 37 } &  0.936712  & 47.200253   &  0.715599   &  20.779371 \\ \hline
\multicolumn{1}{|l|}{Semilla 42 } &  0.874824  & 47.616898   &  0.715599   &  20.961143 \\ \hline
\multicolumn{1}{|l|}{Semilla 440 } &  0.903969  & 47.883408   &  0.715599   &  20.719147 \\ \hline
\multicolumn{1}{|l|}{Semilla 699 } &  0.973378  & 47.606503   &  0.715599   &  21.181446 \\ \hline
\multicolumn{1}{|l|}{Semilla 752 } &  0.920325  & 48.012620   &  0.715599   &  20.775734 \\ \hline
\multicolumn{1}{|l|}{ Media } &  0.921842  & 47.663936   &  0.715599   &  20.883368 \\ \hline
\end{tabular}
      }                
    \end{table} 
    \caption{Función objetivo}
  \end{subfigure}
  \begin{subfigure}{.5\textwidth}
    \centering
    \begin{table}[H]
      \centering
      \resizebox{\textwidth}{!}{
        \begin{tabular}{l|l|l|l|l|}
\cline{2-5}
& \multicolumn{1}{c|}{Iris} & \multicolumn{1}{c|}{Ecoli} & \multicolumn{1}{c|}{Rand} & \multicolumn{1}{c|}{Tyroid}  \\ \hline 
\multicolumn{1}{|l|}{Semilla 37 } &  24.114498  & 249.527510   &  24.124850   &  45.735350 \\ \hline
\multicolumn{1}{|l|}{Semilla 42 } &  24.452518  & 243.258400   &  23.513433   &  44.398865 \\ \hline
\multicolumn{1}{|l|}{Semilla 440 } &  23.470268  & 241.543440   &  23.613468   &  44.316753 \\ \hline
\multicolumn{1}{|l|}{Semilla 699 } &  23.512020  & 241.600170   &  23.420820   &  44.240932 \\ \hline
\multicolumn{1}{|l|}{Semilla 752 } &  23.493227  & 241.310170   &  23.466751   &  44.134445 \\ \hline
\multicolumn{1}{|l|}{ Media } &  23.808506  & 243.447938   &  23.627864   &  44.565269 \\ \hline
\end{tabular}
      }                
    \end{table} 
    \caption{Tiempo de ejecución}
  \end{subfigure}
  \caption{ Resultados para woa con 10\% de restricciones }
  \label{fig:fig}
\end{figure}
\begin{figure}[H]
  \begin{subfigure}{.5\textwidth}
    \centering
    % include first image
    \begin{table}[H]
      \centering
      \resizebox{\textwidth}{!}{
        \begin{tabular}{l|l|l|l|l|}
\cline{2-5}
& \multicolumn{1}{c|}{Iris} & \multicolumn{1}{c|}{Ecoli} & \multicolumn{1}{c|}{Rand} & \multicolumn{1}{c|}{Tyroid}  \\ \hline 
\multicolumn{1}{|l|}{Semilla 37 } &  0.684892  & 12.979752   &  0.715599   &  13.340351 \\ \hline
\multicolumn{1}{|l|}{Semilla 42 } &  0.663448  & 13.042139   &  0.715599   &  13.592555 \\ \hline
\multicolumn{1}{|l|}{Semilla 440 } &  0.662933  & 13.033968   &  0.715599   &  13.620830 \\ \hline
\multicolumn{1}{|l|}{Semilla 699 } &  0.660625  & 13.035360   &  0.715599   &  13.909627 \\ \hline
\multicolumn{1}{|l|}{Semilla 752 } &  0.658393  & 13.160732   &  0.715599   &  13.623517 \\ \hline
\multicolumn{1}{|l|}{ Media } &  0.666058  & 13.050390   &  0.715599   &  13.617376 \\ \hline
\end{tabular}
      }                
    \end{table} 
    \caption{Distancia intracluster}
  \end{subfigure}
  \begin{subfigure}{.5\textwidth}
    \centering
    % include second image
    \begin{table}[H]
      \centering
      \resizebox{\textwidth}{!}{
        \begin{tabular}{l|l|l|l|l|}
\cline{2-5}
& \multicolumn{1}{c|}{Iris} & \multicolumn{1}{c|}{Ecoli} & \multicolumn{1}{c|}{Rand} & \multicolumn{1}{c|}{Tyroid}  \\ \hline 
\multicolumn{1}{|l|}{Semilla 37 } &  77.000000  & 2637.000000   &  0.000000   &  371.000000 \\ \hline
\multicolumn{1}{|l|}{Semilla 42 } &  55.000000  & 2553.000000   &  0.000000   &  294.000000 \\ \hline
\multicolumn{1}{|l|}{Semilla 440 } &  81.000000  & 2663.000000   &  0.000000   &  426.000000 \\ \hline
\multicolumn{1}{|l|}{Semilla 699 } &  76.000000  & 2512.000000   &  0.000000   &  337.000000 \\ \hline
\multicolumn{1}{|l|}{Semilla 752 } &  90.000000  & 2625.000000   &  0.000000   &  348.000000 \\ \hline
\multicolumn{1}{|l|}{ Media } &  75.800000  & 2598.000000   &  0.000000   &  355.200000 \\ \hline
\end{tabular}
      }                
    \end{table}  
    \caption{Infeasibility}
  \end{subfigure}

  \begin{subfigure}{.5\textwidth}
    \centering
    \begin{table}[H]
      \centering
      \resizebox{\textwidth}{!}{
        \begin{tabular}{l|l|l|l|l|}
\cline{2-5}
& \multicolumn{1}{c|}{Iris} & \multicolumn{1}{c|}{Ecoli} & \multicolumn{1}{c|}{Rand} & \multicolumn{1}{c|}{Tyroid}  \\ \hline 
\multicolumn{1}{|l|}{Semilla 37 } &  0.928991  & 48.354507   &  0.715599   &  20.122843 \\ \hline
\multicolumn{1}{|l|}{Semilla 42 } &  0.837804  & 47.290054   &  0.715599   &  18.967363 \\ \hline
\multicolumn{1}{|l|}{Semilla 440 } &  0.919712  & 48.757507   &  0.715599   &  21.408813 \\ \hline
\multicolumn{1}{|l|}{Semilla 699 } &  0.901553  & 46.733270   &  0.715599   &  20.070543 \\ \hline
\multicolumn{1}{|l|}{Semilla 752 } &  0.943702  & 48.374510   &  0.715599   &  19.985538 \\ \hline
\multicolumn{1}{|l|}{ Media } &  0.906353  & 47.901970   &  0.715599   &  20.111020 \\ \hline
\end{tabular}
      }                
    \end{table} 
    \caption{Función objetivo}
  \end{subfigure}
  \begin{subfigure}{.5\textwidth}
    \centering
    \begin{table}[H]
      \centering
      \resizebox{\textwidth}{!}{
        \begin{tabular}{l|l|l|l|l|}
\cline{2-5}
& \multicolumn{1}{c|}{Iris} & \multicolumn{1}{c|}{Ecoli} & \multicolumn{1}{c|}{Rand} & \multicolumn{1}{c|}{Tyroid}  \\ \hline 
\multicolumn{1}{|l|}{Semilla 37 } &  28.139378  & 293.534940   &  28.877132   &  53.587063 \\ \hline
\multicolumn{1}{|l|}{Semilla 42 } &  28.537369  & 284.490780   &  27.549004   &  52.060455 \\ \hline
\multicolumn{1}{|l|}{Semilla 440 } &  27.767319  & 283.893800   &  27.526875   &  52.108093 \\ \hline
\multicolumn{1}{|l|}{Semilla 699 } &  27.650816  & 283.500640   &  27.543203   &  52.094227 \\ \hline
\multicolumn{1}{|l|}{Semilla 752 } &  27.724705  & 282.832120   &  27.512619   &  51.931313 \\ \hline
\multicolumn{1}{|l|}{ Media } &  27.963917  & 285.650456   &  27.801767   &  52.356230 \\ \hline
\end{tabular}
      }                
    \end{table} 
    \caption{Tiempo de ejecución}
  \end{subfigure}
  \caption{ Resultados para woa con 20\% de restricciones }
  \label{fig:fig}
\end{figure}

\section{Análisis de los resultados}

En primer lugar vamos a proceder a comparar los resultados obtenidos con los
resultados usando otros algoritmos 

\begin{table}[H]
  \centering
  \resizebox{\textwidth}{!}{
  \begin{tabular}{|l|l|l|l|l|l|l|l|l|l|l|l|l|}
    \hline
    & \multicolumn{3}{c}{Iris} & \multicolumn{3}{|c}{Ecoli} & \multicolumn{3}{|c}{Rand}& \multicolumn{3}{|c|}{Newthyroid} \\ \hline
    & tasa C  & inf & score  & tasa C & inf & score & tasa C & inf  & score & tasa C & inf & score\\ \hline 
    greedy & 1.800 & 355.8 & 4.057 & 44.613 & 1500.4 & 84.868  & 2.227 & 299.4 & 4.383 &13.160  &958.2 & 48.203\\ \hline
    ls & 0.669 & 0.000  & 0.669 & 21.604 & 92.200 & 24.078 & 0.716  &0.000  &0.716  & 13.246  &24.4 & 14.138\\ \hline
    es & 0.669 & 0.000  & 0.669 & 21.679 & 83.800 & 23.927 & 0.716  & 0.000 & 0.716 & 12.628  & 47.400 & 14.362  \\ \hline
    bmb & 0.669 & 0.000  & 0.669 & 21.997 & 156.000 & 26.183 & 0.716  & 0.000 & 0.716 & 13.835	 & 6.000 & 14.054 \\ \hline
    ils & 0.669 & 0.000  & 0.669 & 21.781 & 75.800 & 23.815	 & 0.716 & 0.000 & 0.716 & 13.835  & 6.000 & 14.054  \\ \hline
    ils-es & 0.669 & 0.000  & 0.669 &  28.198 & 332.800	 & 37.127 & 0.716 & 0.000  & 0.716 &  13.835 & 6.000 & 14.054  \\ \hline
    
    woa & 0.666 & 75.800  & 0.906 &  13.050 & 2598.000 & 47.902 & 0.716 & 0.000  & 0.716 &  13.617 & 355.200 & 20.111  \\ \hline

  \end{tabular}}
  \caption{Resultados medios obtenidos para el problema del PAR con restricciones del 10\%}
\end{table}

%\bibliographystyle{abbrv}
%\bibliography{refs}

\end{document}