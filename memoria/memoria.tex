\documentclass[a4paper,11pt]{article}
\usepackage[spanish]{babel}
\usepackage[utf8]{inputenc}
\usepackage{geometry}
\usepackage{booktabs}  
\usepackage{graphicx} 
\usepackage{listings}
\usepackage{amsmath,amsthm,amssymb}
\usepackage{mathtools}
\usepackage{float}

\usepackage{geometry}
 \geometry{
 a4paper,
 total={170mm,257mm},
 left=20mm,
 top=20mm,
 }

\lstset{%
backgroundcolor=\color{cyan!10},
basicstyle=\ttfamily,
numbers=left,numberstyle=\scriptsize
}

\setlength{\parskip}{\baselineskip}%
\setlength{\parindent}{0pt}%

%\usepackage[wby]{callouts}
\usepackage{hyperref}

\title{Whale Optimziation Algorithm aplicado al problema de clustering}
\author{Yábir García Benchakhtir}
\begin{document}

\maketitle

\begin{figure}[h]
\includegraphics[scale=0.3]{UGR}
\centering
\end{figure}

\newpage

\renewcommand*\contentsname{Índice}
\tableofcontents

\newpage

\section{Introducción al problema}


Consideramos el problema del agrupamiento con restricciones. En este
problema contamos con un conjunto no vacio $P \subset \mathbb{R}^n$ de
puntos y nos planteamos cómo podríamos agruparlos de manera que exista
una relación entre los puntos de un mismo grupo. A cada grupo lo
denominaremos \textit{cluster} y notaremos $\mathcal{C}$ al conjunto
de todos los clusters.

Sobre esta base imponenmos restricciones en la manera en la que se
realizan las agrupaciones. En primer lugar existe un subconjunto de
pares de puntos $ML$ definido como

\[ ML = \{(a,b) \in P\times P\ | \quad a \in K \iff b\in K \text{ para } K \in \mathcal{C}\}\]

es decir, el conjunto de puntos que han de estar en el mismo
cluster. De manera similar existe otro conjunto $CL$ de pares de
puntos que no pueden pertenecer al mismo conjunto.

\[ CL = \{(a,b) \in P\times P\ | \quad a \in K \iff b\notin K \text{ para } K \in \mathcal{C}\}\]

Notaremos por $R = ML \cup CL$ al conjunto de restricciones del problema.

Nos concentraremos en encontrar soluciones bajo restricciones
\textit{fuertes} y restricciones \textit{débiles} a este
problema. Bajo restricciones fuertes todas las restricciones deben
cumplirse y bajo restricciones débiles intentaremos encontrar
soluciones que minimicen el conjunto restricciones violadas.

\section{Whale Optimziation Algorithm}

La metaheurística elegida para resolver el problema ha sido Whale Optimziation
Algorithm (WOA) propuesta por Seyedali Mirjalili y Andrew Lewis en 2016. Esta
metaheurística basa su comportamiento en las técnicas depredadoras de la ballena
jorobada y su comportamiento social. 

Utilizando el especial comportamiento que tiene este animal cuando colabora con 
otros de su misma especia se pretende conseguir una metaheurística que proporcione
buenos resultados en problemas de optimización de funciones reales intentando 
preservar un equilibrio entre exploración y explotación.

Más concretamente la técnica de caza consiste en crear una espiral entorno a la 
presa y levantar un \textit{muro} de burbujas de aire, haciendo que esta se desoriente
para posteriormente acercarse y atacar. 

\subsection{Descripción de la metaheurística}

Para modelar el problema los autores de la metaheurística proponen un modelado 
matemático del comportamiento de la ballena jorobada que se pueda adaptar adaptar 
a la optimización de una función real.

Los agentes $X$ que participan en nuestro algoritmo (y que representan a las
ballenas) se van a representar como $N$ vectores de dimensión $d$ donde cada
componente del vector del agente representa una coordenada en nuestra solución

\[
    X_i(t) = <X_{i1}(t), X_{i2}(t), \dots, X_{id}(t)>
\]

como deja entrever esta notación el estado de la ballena depende de una variable 
temporal $t$ que represneta el instante de tiempo en el que nos encontramos y 
que está limitado por una constante $T$ que fijamos nosotros. En cada instante 
$t$ la mejor solución vendrá represnetada por $X^*$.

El movimiento de caza se represneta por la modificación del vector agente
mediante la expresión

\[
    X_i(t+1) = X^*(t) - A\cdot D_1   
\]

donde $\cdot$ representa el producto componente a componente, $D$ viene dado por
la expresión

\[
    D^1_i = ||CX^*(t)-X_i(t)||    
\]

y $A \text{ y } C$ se obtinen como 

\begin{align*}
    A &= 2a\cdot r - a \\
    C &= 2r
\end{align*}

con $r \in [0,1]^d$ un vector aleatorio y $a\in [0,2]^d$ constante que se hace
decrecer de manera lineal a lo largo de los distintos pasos del algoritmo.

La técnica de \textit{caza} se basa en combinar este movimiento que nos
proporciona un componente de exploración junto a la creación de una espiral 
mediante la actualización del agente de acuerdo a la expresión 

\[
\begin{cases}
    X_i(t+1) &= e^{bl}cos(2\pi l)D_2 \oplus X^* \\     
    D^2_i &= ||X^*(t)-X_i(t)||\\
\end{cases}
\]

siendo $b \in \mathbb{R}$ constante y $l \in [-1,1]$ aleatorio de manera que
nos definen un radio para la espiral en cada instante. La notación $\oplus$ 
se define como 

\begin{align*}
    \oplus\colon & \mathbb{R}\times\mathbb{R}^n \to \mathbb{R}^n\\
    & (a, <x_1, x_2, \dots, x_n>) \xmapsto{}<ax_1, ax_2, \dots, ax_n>
\end{align*}

El movimiento de caza natural mezcla tanto los desplazamientos en linea recta
como el comportamiento en espiral por lo que se introduce un factor aleatorio
$p \in [0,1]$ que decide que tipo de movimeinto se va a realizar

\[
    X_i(t+1) =  
\begin{cases}
    X^*(t) - A\cdot D^1_i & p < \frac{1}{2}\\
    e^{bl}cos(2\pi l)D^2_i + X^*  & p\geq \frac{1}{2}\\  
\end{cases}
\]

\subsection{Adaptación de la metaheurística al problema}

Durante el desarrollo de la asignatura hemos trabajado con una representación
de la solución que se centraba en la asignación de cada punto a un cluster
y se recalculaba en cada caso los centros de cada cluster. El espacio de búsqueda 
era 

\[
    \{<x_1, x_2, \dots x_n >: x_i \in [0, k] \cap \mathbb{N}, k > 0\}
\]

donde $n$ representa el número de puntos que intervienen en el problema y k el número 
de clusters que consideramos.

Para adaptar la metaheurística he decidido variar mi enfoque del problema y, en
lugar de modificar las asignaciones que hago de los puntos, pensar que cada
agente representa las coordenas de los centroiedes del problema.

Así cada agente (ballena) queda definido como 

\[
    X_i = <<c_{01},c_{02}, \dots c_{0d}>, \overset{(k)}{\dots} <c_{k1},c_{k2},\dots c_{kd}>>    
\]

\end{document}